% !Mode:: "TeX:UTF-8"
% !Tex spellcheck = ru-RU
% !TEX  program = xelatex

\documentclass[12pt,a4paper,draft]{article}

\usepackage{csquotes}

\usepackage{polyglossia}
\setdefaultlanguage{russian}
\setotherlanguages{english}

%\defaultfontfeatures{Ligatures={TeX},Renderer=Basic}
\defaultfontfeatures{Ligatures={TeX}}
\setmainfont{Times New Roman}

\usepackage{amsmath}
\newcommand{\angstrom}{\textup{\AA}}

\usepackage[
backend=biber,
style=numeric,
sorting=none
]{biblatex}
\addbibresource{formulas.bib}

\title{Формулы расчёта в jmulti}
\author{С.~В.~Иванов}
\date{январь 2021}

\begin{document}

\maketitle

\tableofcontents

\begin{abstract}
    В этой статье приведена математическая запись формул по которым ведется
    вычисление интенсивности двухволновой дифракции.
\end{abstract}

\section{Атомный фактор}

Для расчёта атомного фактора используются данные из международных
кристаллографических таблиц, том C \cite{itcC}


Из таблицы 4.2.6.8 (\textenglish{Dispersion corrections for forward scattering})
берутся поправочные коэффициенты $f'$ и $f''$. Для промежуточных значений длины
волны используется линейная интерполяция табличных значений.


В зависимости от значения величины $ \sin \theta / \lambda $ для расчёта атомного
фактора использются разные таблицы.
Для интервала $ 0 <= \sin \theta / \lambda <= 2.0 $ используется таблица
6.1.1.4 (\textenglish{Coefficients for analytical approximation to the scattering
factors of Tables 6.1.1.1 and 6.1.1.2}) и формула:

\begin{equation}
    f(\sin \theta  / \lambda) = \sum_{i=1}^{4} a_i \exp(-b_i(\sin \theta / \lambda)^2) + c
\end{equation}

Для интервала $ 2.0 < \sin \theta / \lambda $ используется таблица 6.1.1.5
(\textenglish{Coefficients for analytical approximation to the
scattering factors of Table 6.1.1.1 for the range
$2.0 <\sin \theta / \lambda < 6.0~\angstrom^{-1}$}) и формула:

\begin{equation}
    f(\sin \theta / \lambda) = \sum_{i=0}^{3} a_i (\sin \theta / \lambda)^i
\end{equation}

Величина $ \sin \theta / \lambda $ рассчитывается из условия Брегга по формуле:

\begin{equation}
    \sin \theta / \lambda = \frac{\sqrt{d_{hkl}}}{2},
\end{equation}

где $ d_{hkl} $ обратный квадрат расстояния между кристаллографическими плоскостями соответствующих индексам Миллера $hkl$.

\begin{multline}
    d_{hkl} = \frac{ \left( \frac{h \sin \alpha}{a} \right)^2 +
    \left( \frac{k \sin \beta}{b} \right)^2 +
    \left( \frac{l \sin \gamma}{c} \right)^2 +
    \frac{2kl \cos \alpha}{bc} +
    \frac{2hl \cos \beta}{ac} +
    \frac{2hk \cos \gamma}{ab} }
    {1 - \cos^2 \alpha - \cos^2 \beta -
    \cos^2 \gamma + 2 \cos \alpha \cos \beta \cos \gamma}
\end{multline}

\begin{multline}
    ( \frac{h * h * \sin \alpha * \sin \alpha}{a * a} +
    \frac{k * k * \sin \beta * \sin \beta}{b * b} +
    \frac{l * l * \sin \gamma * \sin \gamma}{c * c} + \\
    \frac{2 * k * l * \cos \alpha}{b * c} +
    \frac{2 * h * l * \cos \beta}{a * c} +
    \frac{2 * h * k * \cos \gamma}{a * b} ) \\ /
        (1 - \cos \alpha * \cos \alpha - \cos \beta * \cos \beta -
        \cos \gamma * \cos \gamma + 2 * \cos \alpha * \cos \beta * \cos \gamma)
\end{multline}

Итоговый атомный фактор это комплексное число образованное суммой $ f + f' +if'' $.

\begin{equation}
    f_a = f + f' + if''
\end{equation}

\section{Структурный фактор}

Структурный фактор отражения от плоскости вычисляется по формуле:

\begin{equation}
    F_{hkl} = \sum_{a} f_a e^{i \pi (hx_a + ky_a + lz_a)}
\end{equation}

где $ x, y, z $ клсталлографические координаты атома в ячейке.

\section{Константы и простые соотношения}

Классический радиус электрона:
\begin{equation}
    r_0 = 2.81794092 * 10^{-5}
\end{equation}

Длина волны из энергии в кЭв:
\begin{equation}
    \lambda = 12.398519 / E
\end{equation}

Угол Брегга основного рефлекса:
\begin{equation}
    \theta_b = \arcsin \left( \frac{\lambda \sqrt{d_{hkl} } }{2} \right)
\end{equation}

Модуль волнового вектора:
\begin{equation}
    k = \frac{2 \pi}{\lambda}
\end{equation}

Учет взаимодействия с магнитным полем зашит в формулы, но поле задано нулевым.

\begin{align}
    F_{Mag} &= 0 + i0 \\
    F_{DQ} &= 0 + i \\
    F_{QQ} &= -1 + i
\end{align}

Объем ячейки и базовые вектора обратной решетки:

\begin{align}
    V &= \vec{a} \cdot (\vec{b} \times \vec{c}) \\
    \vec{a}_{Rec} &= \frac{2 \pi (\vec{b} \times \vec{c})}{V} \\
    \vec{b}_{Rec} &= \frac{2 \pi (\vec{c} \times \vec{a})}{V} \\
    \vec{c}_{Rec} &= \frac{2 \pi (\vec{a} \times \vec{b})}{V}
\end{align}

\begin{equation}
    \chi_0 = \frac{-4 \pi r_0 \sum_{a} Z_a + f' + if''}{k^2 V}
\end{equation}

\section{Отбор пар плоскостей}

Перед рассчетом интенсивности рефлекса производится выборка
пар прлоскостей ( $abc$ плоскость первого отражения, $a'b'c'$
плоскость второго отражения). При этом из условия что нужны только такие пары,
которые дают рефлекс в том же направлениии, что и основной от плоскости $hkl$,
следует, что индексы $abc$ и $a'b'c'$ связаны соотношением:

\begin{equation}
    a' = h - a, b' = k - b, c' = l - c .
\end{equation}

Таким образом выбор первой плоскости однозначно определяет и вторую плоскость.

Также производится вычисление структурного фактора двойного
отражения $ F_{hkl}^{abc} $, тут верхние индексы $abc$ использованы
для обозначения для какой пары плоскостей вычисляется этот структурный фактор.

\begin{equation}
    F_{hkl}^{abc} = F_{abc} F_{a'b'c'}
\end{equation}

Выборка плоскостей осуществляется пребором индексов $abc$ в заданных пределах,
в выборку попадают только плоскости удовлетворяюще условиям:

\begin{align}
    \frac{\sqrt{d_{abc}}}{2} &<= \frac{5}{\lambda} \\
    \frac{\sqrt{d_{a'b'c'}}}{2} &<= \frac{5}{\lambda} \\
    \lvert F_{abc} \rvert &>= 10^{-6} \\
    \lvert F_{a'b'c'} \rvert &>= 10^{-6} \\
    \lvert F_{hkl}^{abc} \rvert &>= 1
\end{align}

\section{Расчет интенсивности рефлексов}

В расчете для упрощения формул используется локальные декартовы координаты
с базовыми векторами $\vec{a_0}, \vec{b_0}, \vec{n}$. Выбор этих векторов зависит
от параметров ячейки и основного рефлекса $hkl$. Подробности выбора векторов опустим,
скажем только, что $\vec{n}$ совпадает с вектором нормали плоскости $hkl$,
a угол $\psi$ отсчитывается от вектора $\vec{a_0}$ в направлении $\vec{b_0}$.

Волновой вектор $\vec{k}$ и вектора поляризации падающего излучения
$\vec{e_s}, \vec{e_p}$ и после рассеяного $\vec{e_p'}$:

\begin{align}
    \vec{k} &= k \left( \cos \theta_b \cos \psi \vec{a_0} + \cos \theta_b \sin \psi \vec{b_0} +- \sin \theta_b \vec{n} \right) \\
    \vec{e_s} &= - \sin \psi \vec{a_0} + \cos \psi \vec{b_0} \\
    \vec{e_p} &= \sin \theta_b \cos \psi \vec{a_0} + \sin \theta_b \sin \psi \vec{b_0} + \cos \theta_b \\
    \vec{e_p'} &= - \sin \theta_b \cos \psi \vec{a_0} - \sin \theta_b \sin \psi \vec{b_0} + \cos \theta_b \vec{n}
\end{align}

Здесь $a, b, c$ это индексы отобранных на предыдущем шаге прлоскостей.

\begin{align}
    \vec{p} &= a \vec{a}_{Rec} + b \vec{b}_{Rec} + c \vec{c}_{Rec} \\
    \vec{k}_n &= \vec{k} + \vec{p} \\
    \mathbf{k}_n^2 & = \lvert \vec{k} + \vec{p} \rvert \\
    k_{ns} &= \vec{k}_n \cdot \vec{e_s} \\
    k_{np} &= \vec{k}_n \cdot \vec{e_p} \\
    k_{np}' &= \vec{k}_n \cdot \vec{e_p'} \\
    F_{Mult}^{ss} &= \frac{4 \pi r_0}{k^2 V} \sum_{abc} \frac{F_{hkl}^{abc} (\vec{k}_n^2 - k_{ns}^2)}{\vec{k}_n^2 (1 - \chi_0) - k^2} \\
    F_{Mult}^{pp} &= \frac{4 \pi r_0}{k^2 V} \sum_{abc} \frac{F_{hkl}^{abc} (\vec{k}_n^2 \cos 2\theta_b - k_{np}k_{np}')}{\vec{k}_n^2 (1 - \chi_0) - k^2} \\
    F_{Mult}^{ps} &= - \frac{4 \pi r_0}{k^2 V} \sum_{abc} \frac{F_{hkl}^{abc} k_{np}'k_{ns}}{\vec{k}_n^2 (1 - \chi_0) - k^2} \\
    F_{Mult}^{sp} &= - \frac{4 \pi r_0}{k^2 V} \sum_{abc} \frac{F_{hkl}^{abc} k_{ns}k_{np}}{\vec{k}_n^2 (1 - \chi_0) - k^2} \\
\end{align}

\begin{align}
    c_{QQ} &= F_{QQ} \sin 3 \psi \\
    \lvert F_{Mod}^{ss} \rvert &= \lvert F_{Mult}^{ss} + F_{Mag} \rvert \\
    \lvert F_{Mod}^{pp} \rvert &= \lvert F_{Mult}^{pp} + F_{Mag} \rvert \\
    \lvert F_{Mod}^{ps} \rvert &= \lvert F_{Mult}^{ps} - F_{Mag} + c_{QQ} + F_{DQ} \rvert \\
    \lvert F_{Mod}^{sp} \rvert &= \lvert F_{Mult}^{sp} + F_{Mag} + c_{QQ} - F_{DQ} \rvert \\
    F_{Mod}^s &= \lvert F_{Mod}^{ps} \rvert^2 + \lvert F_{Mod}^{ss} \rvert^2 \\
    F_{Mod}^p &= \lvert F_{Mod}^{pp} \rvert^2 + \lvert F_{Mod}^{sp} \rvert^2 \\
    Rr2 &= \frac{\lvert F_{Mult}^{ss} + i F_{Mult}^{sp} - i F_{Mag} \rvert^2 + \lvert i F_{Mult}^{pp} + F_{Mult}^{ps} + F_{Mag}} \rvert^2{2} \\
    Rl2 &= \frac{\lvert F_{Mult}^{ss} - i F_{Mult}^{sp} + i F_{Mag} \rvert^2 + \lvert -i F_{Mult}^{pp} + F_{Mult}^{ps} + F_{Mag} \rvert^2}{2}
\end{align}

\printbibliography

\end{document}

