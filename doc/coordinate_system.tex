%%% Local Variables:
%%% coding: utf-8
%%% mode: latex
%%% TeX-engine: xetex
%%% End:

% !Mode:: "TeX:UTF-8"
% !TeX spellcheck = ru-RU
% !TeX program = xelatex

\documentclass[12pt,a4paper,draft]{article}

\usepackage{csquotes}

\usepackage{polyglossia}
\setdefaultlanguage{russian}
\setotherlanguages{english,german}

\defaultfontfeatures{Ligatures={TeX},Renderer=Basic}
\setmainfont{Times New Roman}

\usepackage{amsmath}

\usepackage[
backend=biber,
style=numeric,
sorting=none
]{biblatex}
\addbibresource{structures.bib}

\title{Система координат jmulti}
\author{С.~В.~Иванов}
\date{июнь-сентябрь 2020}

\begin{document}

\maketitle

\tableofcontents

\begin{abstract}
  На разных этапах вычислений в программе jmulti используются три системы
  отсчёта: система координат ячейки $(a, b, c)$,
  система координат эксперимента $(\psi, \theta)$ и декартовы координаты
  $(x, y, z)$. Чтобы правильно интерпретировать
  результаты работы программы необходимо понимать как эти системы координат
  связаны между собой.

  Декартовы координаты используются как базовая система отсчёта,
  расчёт интенсивности рефлекса ведётся в ней.

  Система координат ячейки располагается относительно декартовой следующим
  образом: ось $\vec{a}$ совмещается с осью $\vec{x}$,
  а ось $\vec{b}$ помещается в плоскости осей $\vec{x}, \vec{y}$, при этом
  плоскость $xy$ системы совмещается с плоскостью $ab$ ячейки. Таким образом в
  случае прямоугольной ячейки направления осей $\vec{a}$, $\vec{b}$
  и $\vec{c}$ совпадают с направлениями осей $\vec{x}$, $\vec{y}$ и
  $\vec{z}$, соответственно.

  В системе координат эксперимента $\theta$ это угол Брэгга, который
  отсчитывается от плоскости $\mathbf{P}$ рефлекса $(hkl)$. А $\psi$ это
  азимутальный угол поворота вокруг оси перпендикулярной плоскости $\mathbf{P}$.
  Для определения точки отсчёта $\psi$ формируется промежуточная декартова
  система координат $(\vec{x'}, \vec{y'}, \vec{z'})$,
  угол отсчитывается от вектора $\vec{x'}$ в направлении к $\vec{y'}$.
  
  Базовые вектора этой промежуточной системы координат ориентируются так:
  \begin{itemize}
    \item Вектор $\vec{z'}$ направлен параллельно
      вектору нормали плоскости $\mathbf{P}$.
    \item Вектор $\vec{x'}$ направлен параллельно проекции
      одного из базовых векторов ячейки на плоскость $\mathbf{P}$
      в зависимости от индексов $(hkl)$.
    \item Вектор $\vec{y'}$ дополняет вектора $\vec{x'}$ и $\vec{z'}$
      таким образом, чтобы тройка $(\vec{x'}, \vec{y'} \vec{z'})$
      была правой.
  \end{itemize}

\end{abstract}

\section{Декартовы координаты $(\vec{x}, \vec{y}, \vec{z})$}
  Базовая система отсчёта, расчёт интенсивности рефлекса ведётся в ней.

\section{Координаты кристаллической ячейки $(\vec{a}, \vec{b}, \vec{c})$}
  Система координат ячейки располагается относительно декартовой следующим
  образом: ось $\vec{a}$ совмещается с осью $\vec{x}$,
  а ось $\vec{b}$ помещается в плоскости осей $\vec{x}$, $\vec{y}$.
  Таким образом в случае прямоугольной ячейки направления
  осей $\vec{a}$, $\vec{b}$ и $\vec{c}$ совпадают с направлениями
  осей $\vec{x}$, $\vec{y}$ и $\vec{z}$, соответственно.

  Программа jmulti считывает параметры ячейки
  ($a$, $b$, $c$, $\alpha$, $\beta$, $\gamma$) из cif файла и позволяет
  их отредактировать перед запуском расчёта.
  Никаких проверок корректности этих параметров не производится.

  Базовые вектора ячейки $(\vec{a}, \vec{b}, \vec{c})$ в декартовой системе
  координат записываются по следующей формуле:
  \begin{align}
    \vec{a} &= \begin{bmatrix} a & 0 & 0 \end{bmatrix} \\
    \vec{b} &= \begin{bmatrix} b \cos\gamma & b \sin\gamma & 0 \end{bmatrix} \\
    cz &= \dfrac{c \sqrt{1 - \cos^2\alpha - \cos^2\beta - \cos^2\gamma + 2 \cos\alpha \cos\beta \cos\gamma}}{\sin\gamma} \nonumber \\
    \vec{c} &= \begin{bmatrix} c \cos\beta & \dfrac{c (\cos\alpha - \cos\beta \cos\gamma)}{\sin\gamma} & cz \end{bmatrix}
  \end{align}

  Таким образом для кварца\cite{quartz} с параметрами ячейки
  $a = b = 4.91 Å$, $c = 5.4 Å$, $\alpha = \beta = 90°$, $\gamma = 120°$
  представление базовых векторов ячейки в декартовых координатах
  будет выглядеть следующим образом:
  \begin{align}
    \vec{a} &= \begin{bmatrix} 4.91 & 0 & 0 \end{bmatrix} \nonumber \\
    \vec{b} &= \begin{bmatrix} -2.455 & 4.252 & 0 \end{bmatrix} \nonumber \\
    \vec{c} &= \begin{bmatrix} 0 & 0 & 5.4 \end{bmatrix} \nonumber
  \end{align}

  А для микроклина\cite{microcline}
  $a = 8.56 Å$, $ b = 12.964 Å$, $c = 7.215 Å$,
  $\alpha = 90.65°$, $\beta = 115.83°$, $\gamma = 87.7°$:
  \begin{align}
    \vec{a} &= \begin{bmatrix} 8.56 & 0 & 0 \end{bmatrix} \nonumber \\
    \vec{b} &= \begin{bmatrix} 0.520 & 12.954 & 0 \end{bmatrix} \nonumber \\
    \vec{c} &= \begin{bmatrix} -3.144 & 0.044 & 6.494 \end{bmatrix} \nonumber
  \end{align}


\section{Система координат эксперимента $(\psi, \theta)$}
  $\theta$ это угол Брэгга, который отсчитывается от плоскости $\mathbf{P}$
  рефлекса $(hkl)$.
  $\psi$ это азимутальный угол поворота вокруг оси, перпендикулярной
  плоскости $\mathbf{P}$.
  Для определения начала отсчёта угла $\psi$ формируется промежуточная
  система отсчёта $(\vec{x'}, \vec{y'}, \vec{z'})$
  угол $\psi$ отсчитывается от вектора $\vec{x'}$ в направлении к $\vec{y'}$
  этой системы отсчёта.
  
  Базис этой промежуточной системы выбирается следующим образом:
  \begin{itemize}
    \item Направление вектора $\vec{z'}$ совпадает с направлением вектора
      нормали плоскости $\mathbf{P}$.
    \item В зависимости от соотношения углов между вектором $\vec{z'}$
      и базовыми векторами ячейки $\vec{a}$, $\vec{b}$ и $\vec{c}$ выбирается
      вектор $\vec{x'}$. Если наименьший из трёх это угол между $\vec{z'}$
      и $\vec{a}$, то качестве направления вектора $\vec{x'}$ задаётся
      направление проекции вектора $\vec{b}$ на плоскость $\mathbf{P}$.
      Если наименьший угол между $\vec{z'}$ и $\vec{b}$, то используется
      проекция вектора $\vec{c}$. Во всех остальных случаях используется
      проекция вектора $\vec{a}$.
    \item Вектор $\vec{y'}$ дополняет вектора $\vec{x'}$ и $\vec{z'}$
      до правой тройки.
  \end{itemize}
  В результате вектора $\vec{x'}$ и $\vec{y'}$ лежат в плоскости $\mathbf{P}$,
  а вектор $\vec{z'}$ перпендикулярен $\mathbf{P}$ и совпадает с
  вектором нормали. Угол $\psi$ отсчитывается от вектора $\vec{x'}$
  по направлению к вектору $\vec{y'}$.

\printbibliography

\end{document}
